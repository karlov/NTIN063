\section{\texorpdfstring{Polynomial Hierarchy}{Polynomial Hierarchy}}
\vspace{5mm}
\large

\subsection{Time and Space classes relation}

\begin{reminder}[Space classes]
	\begin{equation}
	\begin{split}
		LOG = DS(\log n) \\
		NLOG = NS(\log n) \\
		POLYLOG = \bigcup_{i \geq 0} DS(\log^i n )\\
		PS = \bigcup_{i \geq 0} DS(n^i)\\
		NSSP = \bigcup_{i \geq 0} NS(n^i)\\
		EXPSPACE = \bigcup_{i \geq 0} DS(2^{ni})\\
	\end{split}
	\end{equation}
\end{reminder}

\begin{reminder}[Time classes]
	No way to define LOG class, as we have to read the input.
	\begin{equation}
	\begin{split}
		\TP = \bigcup_{i \geq 0} DT(n^i)\\
		\TNP = \bigcup_{i \geq 0} NT(n^i)\\
		DEXT = \bigcup_{i \geq 0} DT(2^{ni})\\
		NEXT = \bigcup_{i \geq 0} NT(2^{ni})\\
		EXPTIME = \bigcup_{i \geq 0} DT(2^{n^i})\\
		NEXPTIME = \bigcup_{i \geq 0} NT(2^{n^i})\\
	\end{split}
	\end{equation}
\end{reminder}

\begin{definition}
	Consider a 3 sequences of classes
	\[ \Sigma_k, \Pi_k, \Delta_k \]
	Where
	\begin{enumerate}
		\item $\Sigma_0 = \Pi_0 = \Delta_0 = \TP$.
		\item $\Sigma_{k + 1} = \TNP(\Sigma_k)$.
		\item $\Pi_{k + 1} = co-\TNP(\Sigma_k)$.
		\item $\Delta_{k + 1} = \TP(\Sigma_k)$.
	\end{enumerate}

	And Polynomial hierarchy is:
	\[ PH = \bigcup_{i \geq 1} \Sigma_i^P ( = \bigcup_{i \geq 1} \Pi_i^P = \bigcup_{i \geq 1} \Delta_i^P) \]

\end{definition}

\begin{note}
	L is a language alphabet $\tau$
	\[ \overline{L} = \{ x \in \tau^{\ast} |\ x \notin L \} \]

	If C is a class of languages, then
	\[ L \in C \iff \overline{L} \in co-C \]
\end{note}

\begin{properties}[Polynomial Hierarchy].
	\begin{enumerate}[label=\alph*)]
		\item $\Sigma_1 = \TNP $.
		\item $\Pi_k = co-\Sigma_k \land \Sigma_k = co-\Pi_k$.
		\item $\Sigma_{k + 1} = \TNP(\Pi_k)$.
		\item $\Delta_{k + 1} = \TP(\Pi_k)$.
		\item $\Pi_{k + 1} = co-\TNP(\Pi_k)$.
		\item $\Sigma_{k + 1} = \TNP(\Delta_{k + 1})$.
		\item $\Pi_{k + 1} = co-\TNP(\Delta_{k + 1})$.
	\end{enumerate}
\end{properties}
\begin{proof}
	a) $\TNP(\TP) = \TNP$ also $\TP(\TP) = \TP$.\\
	Proof by embedding oracle computation by simulating using DTM.
\end{proof}

\begin{properties}[Polynomial Hierarchy - 2].
	%\begin{enumerate}[label=\alph*)]
	%\end{enumerate}
\end{properties}
\begin{proof}
\end{proof}
