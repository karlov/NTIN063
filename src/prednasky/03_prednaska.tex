\section{\texorpdfstring{Polynomial Hierarchy}{Polynomial Hierarchy}}
\vspace{5mm}
\large

\subsection{Simplified PH}

\begin{definition}
	Consider a sequence
	\[ \Sigma_0^P, \Sigma_1^P, \Sigma_2^P, \ldots \]
	Where $\Sigma_0^P = \TP, \Sigma_{i + 1}^P = \TNP(\Sigma_i^P)$.

	And Polynomial hierarchy(simplified) is:
	\[ PH = \bigcup_{i \geq 1} \Sigma_i^P \]
\end{definition}

\begin{theorem}[Polynomial hierarchy(simplified)]
	\[ PH \subseteq PS \]

	In plain words, PH is only smth in between NP and PS.
\end{theorem}
\begin{proof}
	By induction prove $\forall i: \Sigma_i \subseteq PS$.

	\begin{itemize}
		\item $i = 0 \Rightarrow \TP \subseteq PS$
		\item $i \to i + 1$, assume $\Sigma_i \subseteq PS$.
	\end{itemize}

	By definition: $\Sigma_{i + 1} = \TNP(\Sigma_i)$
	Then
	\[ \TNP(\Sigma_i) \subseteq \TNP(PS) \]
	We made set of oracles larger, set of recognized languages cannot shrink.

	Now we use $\forall C: \TNP(C) \subseteq PS(C)$.\\
	Therefore
	\[ \TNP(C) \subseteq PS(PS) \subseteq PS \]

	Last inclusion is up to prove (similar to $\TP = \TP(\TP)$):\\
	\[ B \in PS(PS) \iff \exists A \in PS \exists DTM M: B = L(M, A) \]
	Where $M$ works in poly space.

	\[ A \in PS \iff \exists DTM \ M_A: A = L(M_A) \]
	Where $M_A$ works in poly space.
	Same in $\TP = \TP(\TP)$ proof replace oracle QUERY by DTM computation (which accepts or rejects)

	The last thing is to check that used space is polynomial.
	Which is true since $\forall t \in QUERY: |t| \leq p(|x|)$ for some polynomial $p$.
	Space taken by DTM $M^{\prime}$ that computes the QUERY is $p_t(|t|) \leq p_t(p(|x|))$.
	Also, we can ask exponentially many QUERies, however the space is reused, therefore space is bounded by the largest QUERY.
\end{proof}

\subsection{Time and Space classes relation}

\begin{reminder}[Space classes]
	\begin{equation}
	\begin{split}
		LOG = DS(\log n) \\
		NLOG = NS(\log n) \\
		POLYLOG = \bigcup_{i \geq 0} DS(\log^i n )\\
		PS = \bigcup_{i \geq 0} DS(n^i)\\
		NSSP = \bigcup_{i \geq 0} NS(n^i)\\
		EXPSPACE = \bigcup_{i \geq 0} DS(2^{ni})\\
	\end{split}
	\end{equation}
\end{reminder}

\begin{reminder}[Time classes]
	No way to define LOG class, as we have to read the input.
	\begin{equation}
	\begin{split}
		\TP = \bigcup_{i \geq 0} DT(n^i)\\
		\TNP = \bigcup_{i \geq 0} NT(n^i)\\
		DEXT = \bigcup_{i \geq 0} DT(2^{ni})\\
		NEXT = \bigcup_{i \geq 0} NT(2^{ni})\\
		EXPTIME = \bigcup_{i \geq 0} DT(2^{n^i})\\
		NEXPTIME = \bigcup_{i \geq 0} NT(2^{n^i})\\
	\end{split}
	\end{equation}
\end{reminder}

\begin{exercise}
	Complexity classes
	\begin{enumerate}[label=\alph*)]
		\item $NLOG \subseteq \TP$
		\item $PS = NPS$
		\item $\TNP \subseteq PS$
		\item $PS = EXPTIME$
		\item $NLOG \subsetneq PS \subsetneq EXPSPACE$
		\item $\TP \subsetneq DEXT \subsetneq EXPTIME$
	\end{enumerate}

\end{exercise}
\begin{proof}
	a)
	\[ L \in NS(\log n) \Rightarrow \exists c_L: L \in DT(2^{c_L \log n}) = DT((2^{\log n})^{c_L}) = DT(n^{c_L}) \in \TP \]
	b) $PS \subseteq NPS$ trivial, as deterministic computation is a special case of non-deterministic.

	$NPS \subseteq PS$
	\[ L \in NS \Rightarrow \exists i: L \in NS(n^i) \]
	by Savic \cref{savic}
	\[ L \in DS(n^{2i}) \Rightarrow L \in PS \]

	c) we use time space relation \cref{nt_ds_rel}
	\[ \forall L \in \TNP \Rightarrow \exists i: L \in NT(n^i) \Rightarrow L \in DS(n^i) \Rightarrow \TNP \subseteq PS \]
	d)
	\[ L \in PS \Rightarrow \exists i: L \in DS(n^i) \subseteq NS(n^i) \Rightarrow \exists c_L: L \in DT(2^{c_L \log n}) \subseteq DT(2^{n^{i + 1}}) \subseteq EXPTIME \]
	e) by Savic \cref{savic}
	\[ NS(\log n) \subseteq DS(\log^2 n) \]
	by space hierarchy \cref{s_hier}
	\[ \log^2 n \in o(n) \Rightarrow DS(\log^2 n) \subsetneq DS(n) \]
	\[ L \in PS \Rightarrow \exists i: L \in DS(n^i) \subseteq DS(2^n) \]
	by space hierarchy \cref{s_hier}
	\[ DS(2^n) \subsetneq DS(2^{2n}) \]

	f)
	by time hierarchy \cref{t_hier}
	\[ \TP \subseteq DT(2^n) \subsetneq DT(2^{2n}) \subseteq DEXT \]
	as
	\[ 2^n \log(2^n) = n 2^n \in o((2^n)^2) \]
	Then
	\[ DEXT \subseteq DT(2^{n^2}) \subsetneq DT(2^{n^3}) \subseteq EXPTIME \]
	as
	\[ n^2 (2^{n^2}) \in o (2^{n^3}) \]
\end{proof}

\begin{note}
	\[ NLOG \subseteq \TP \subseteq \TNP \subseteq PS \]
	Also
	\[ NLOG \subsetneq PS \]
	Therefore one of the inclusions in first relation should be strict, or all of them. Still open question.

	Moreover, if we prove strict equalities in leftmost and rightmost inclusions $\Rightarrow \TP \neq \TNP$.
\end{note}

\subsection{Full Polynomial hierarchy}

\begin{definition}
	Consider a 3 sequences of classes
	\[ \Sigma_k, \Pi_k, \Delta_k \]
	Where
	\begin{enumerate}
		\item $\Sigma_0 = \Pi_0 = \Delta_0 = \TP$.
		\item $\Sigma_{k + 1} = \TNP(\Sigma_k)$.
		\item $\Pi_{k + 1} = co-\TNP(\Sigma_k)$.
		\item $\Delta_{k + 1} = \TP(\Sigma_k)$.
	\end{enumerate}

	And Polynomial hierarchy is:
	\[ PH = \bigcup_{i \geq 1} \Sigma_i^P ( = \bigcup_{i \geq 1} \Pi_i^P = \bigcup_{i \geq 1} \Delta_i^P) \]

\end{definition}

\begin{note}
	L is a language alphabet $\tau$
	\[ \overline{L} = \{ x \in \tau^{\ast} |\ x \notin L \} \]

	If C is a class of languages, then
	\[ L \in C \iff \overline{L} \in co-C \]
\end{note}

\begin{properties}[Polynomial Hierarchy] Relations
	\begin{enumerate}[label=\alph*)]
		\item $\Sigma_1 = \TNP $.
		\item $\Pi_k = co-\Sigma_k \land \Sigma_k = co-\Pi_k$.
		\item $\Sigma_{k + 1} = \TNP(\Pi_k)$.
		\item $\Delta_{k + 1} = \TP(\Pi_k)$.
		\item $\Pi_{k + 1} = co-\TNP(\Pi_k)$.
		\item $\Sigma_{k + 1} = \TNP(\Delta_{k + 1})$.
		\item $\Pi_{k + 1} = co-\TNP(\Delta_{k + 1})$.
	\end{enumerate}
\end{properties}
\begin{proof}
	a) $\TNP(\TP) = \TNP$ also $\TP(\TP) = \TP$.\\
	Proof by embedding oracle computation by simulating using DTM.

	b) by definition, for $k \geq 1$
	\[ \Pi_k = co-\TNP(\Sigma_{k - 1}) = co-\Sigma_k \]
	same for $\Sigma_k$ by symmetry.

	c) we can deduce $\overline{B}$ from questions to $B$ by negating every request.

	% todo d), e)

	f)
	\[ \Sigma_{k + 1} = \TNP(\Sigma_k) = \TNP(\Delta_{k - 1}) = \TNP(\TP(\Sigma_k)) = \TNP(\Delta_k) \]
	Clearly
	\[ \TNP(\Sigma_k) \subseteq \TNP(\TP(\Sigma_k)) \]
	as we made class of query languages larger.
	\[ L \in \TNP(\TP(\Sigma_k)) \Rightarrow \exists NTM\ M_n \exists E \in \TP(\Sigma_k): L = L(M_n, E) \]
	Also
	\[ E \in \TP(\Sigma_k) \iff \exists DTM\ M_d, \exists Q \in \Sigma_k: E = L(M_d, Q) \]
	We want $L \in \TNP(\Sigma_k)$.
	We proceed similarly as in $\TP(\TP) = \TP$ by simulating $M_n$ and every time it ask a query, use $M_d$ to decide.

	% todo
	g)

\end{proof}

\begin{properties}[Polynomial Hierarchy - 2] Relations 2.
	\begin{enumerate}[label=\alph*)]
		\item $\Delta_k = co-\Delta_k$
		\item $\TP(\Delta_k) = \Delta_k$
		\item $\Sigma_k \cup \Pi_k = \Delta_{k + 1}$
		\item $\Delta_k = \Sigma_k \cap \Pi_k$
		\item if $\Sigma_k \subseteq \Pi_k \lor \Pi_k \subseteq \Sigma_k \Rightarrow \Pi_k = \Sigma_k$
	\end{enumerate}
\end{properties}
\begin{proof}
	a) for deterministic computation we can negate every answer.
	Specifically this rule means $\TP = co-\TP$

	b) $\Delta_k \subseteq \TP(\Delta_k)$ trivially. Since we can use same language as oracle with 1 query??

	$\TP(\Delta_k) \subseteq \Delta_k$. Proof by induction on $k$:
	\[ k = 0, \TP(\TP) = \TP \]

	$k \geq 1$ we have $\TP(\Delta_{k - 1}) \subseteq \Delta_{k - 1}$
	By definition $\TP(\Delta_k) \subseteq \Delta_k$ is equivalent to
	\[ \TP(\TP(\Delta_{k - 1})) \subseteq \TP(\Delta_{k - 1}) \]

	Both c) and d) use the fact, that deterministic computation is a special case of non-deterministic.

	c) proof consists of 2 steps
	\begin{enumerate}
		\item $\Sigma_k \subseteq \Delta_{k + 1} = \TP(\Sigma_k)$\\
			Same argument as in first inclusion in b).
			Ask single query to the same language.
		\item $\Pi_k \subseteq \Delta_{k + 1} = \TP(\Sigma_k)$\\
			Since we can negate queries
			\[ \TP(\Sigma_k) = \TP(\Pi_k) \]
			Therefore using same argument as above with single query
			\[ \Pi_k \subseteq \TP(\Pi_k) \]
	\end{enumerate}

	d) proof consists of 2 steps
	\begin{enumerate}
		\item TODO check first inclusion
			\[ \Delta_k = \TP(\Sigma_{k - 1}) \subseteq \TNP(\Sigma_{k - 1}) = \Sigma_k \]
		\item
			\[ \Delta_k \stackrel{\text{a)}}{=} co-\Delta_k = co-\TP(\Sigma_{k - 1}) = co-\TP(\Pi_{k - 1}) \subseteq co-\TNP(\Pi_{k - 1}) = \Pi_k \]
	\end{enumerate}

	e) Assume $\Sigma_k \subseteq \Pi_k$ we need to prove reverse.

	\[ L \in \Pi_k \iff \overline{L} \in co-\Pi_k = \Sigma_k \xRightarrow{assumption} \overline{L} \in co-\Sigma_k = \Pi_k \iff L \in \Sigma_k  \]
\end{proof}

\begin{theorem}[NP = co-NP]
	if $A \in \TNP$-complete $\land A \in co-\TNP$-complete $\Rightarrow \TNP \subseteq co-\TNP$.
\end{theorem}
\begin{proof}
	Let $B \in \TNP$ arbitrary.
	By the $\TNP$-completeness of A $\exists DTM$ transducer $M_t$ st
	\[ x \in B \iff M_t(x) \in A \]

	\[ A \in co-\TNP \iff \overline{A} \in \TNP \Rightarrow \exists NTM\ M_n: \overline{A} = L(M_n) \]
	Also by the properties of polynomial reduction
	\[ x \in \overline{B} \iff M_t(x) \in \overline{A} \]
	$\overline{A}$ can be recognized by $M_n$, therefore we can recognize $\overline{B}$ by NTM which is a concatenation of $M_t$ and $M_n$.
	Therefore $\overline{B} \in \TNP \iff B \in co-\TNP $.

	Same proof is valid for oracle TM.
\end{proof}

\begin{theorem}[NP $\neq$ co-NP]
	\[ \TNP \cap\ co-\TNP \Rightarrow \TNP \cup\ co-\TNP \subsetneq \Delta_2 \]
\end{theorem}
\begin{proof}
	Let
	\[ L = \{ (F, F^{\prime} |\ F \in SAT \land F^{\prime} \in \overline{SAT} \} \]
	We can construct DTM with following algorithm for L

	%todo add identation for readbility
	\begin{enumerate}
		\item if($F \in SAT$)
		\item \tab if($\overline{F} \in SAT$)
		\item \tab \tab accept
		\item \tab else
		\item \tab \tab reject
		\item else
		\item \tab reject
	\end{enumerate}

	Therefore $L \in \TP(SAT) \subseteq \Delta_2 = \TP(\Sigma_1) = \TP(\Pi_k)$.

	Define
	\[ L_2 = \{ (F, 0)|\ F \in SAT \} \subseteq L \]
	where $0$ is e.g. $(x \land \neg x)$.

	Trivially $ L_2 \simeq SAT$ therefore L contains $\TNP$-complete language.

	Now assume by contradiction $L \in co-\TNP$.
	Using previous proof we can deduce $\TNP = co-\TNP$.
\end{proof}

\subsection{Constrained quantifiers}
\begin{definition}
	\[ \exists^{p(n)} x: R(x) := \exists x: |x| \leq p(n) \land R(x) \]
\end{definition}
\begin{definition}
	\[ \forall^{p(n)} x: R(x) := \forall x: |x| \leq p(n) \land R(x) \]
\end{definition}

\begin{definition}
	Let $C$ be a class of languages, we define class of languages $\exists C$ as following:
	\[ A \in (\exists C) \xLeftrightarrow{def} \exists B \in C, \exists p: x \in A \iff \exists^{p(|x|)} y: \langle x, y \rangle \in B \]

	Note that if $C = \TP \Rightarrow \exists \TP = \TNP $.
	Since $y$ is a branch of NTM (certificate)
\end{definition}

\begin{definition}
	Let $C$ be a class of languages, we define class of languages $\forall C$ as following:
	\[ A \in (\forall C) \xLeftrightarrow{def} \exists B \in C, \forall p: x \in A \iff \forall^{p(|x|)} y: \langle x, y \rangle \in B \]

	Note that if $C = \TP \Rightarrow \forall \TP = co-\TNP $.
\end{definition}

\begin{lemma}\label{quant_class_rel}
	Let $C$ be an arbitrary class of languages. Then
	\begin{equation}
		co-\exists C = \forall(co-C)
	\end{equation}
	As
	\[ A \in (\exists C) \iff \overline{A} \in (\forall(co-C)) \]

	\begin{equation}
		C \subseteq \exists C \land C \subseteq \forall C
	\end{equation}
\end{lemma}
\begin{proof}
	Equation (3): \\
	Negating the definition
	\[ A \in (\exists C) \iff \exists B \in C, \exists p: x \in A \iff \exists^{p(|x|)} y: \langle x, y \rangle \in B \]
	we get
	\[ x \in \overline{A} \iff \forall^{p(|x|)} y: \langle x, y \rangle \in \overline{B} \]
	Which is the same as
	\[ \overline{A} \in \forall (co-C) \iff \exists \overline{B} \in co-C, \exists p: \forall^{p(|x|)} y: \langle x, y \rangle \in \overline{B} \]

	Equation (4): \\
	For $\exists C$ take $B = A \land y = \emptyset$.
	Clearly $\langle x, \emptyset \rangle \simeq x$.

	For $\forall C$ take polynomial with degree 0, only $y = \emptyset$ is accepted.
\end{proof}

\begin{properties}[Polynomial Hierarchy with quantifiers]\label{polyn_q}
	\begin{enumerate}[label=\alph*)]
		\item $\exists \TP = \TNP$
		\item $\forall \TP = co-\TNP $.
		\item $\forall k > 0: \exists \Sigma_k = \Sigma_k$.
		\item $\forall k > 0: \forall \Pi_k= \Pi_k$.
		\item $\forall k \geq 0: \exists \Pi_k = \Sigma_{k + 1}$.
		\item $\forall k \geq 0: \forall \Sigma_k = \Pi_{k + 1}$.
	\end{enumerate}
\end{properties}
\begin{proof}
	a) "$\exists \TP \subseteq \TNP$"
	\[ A \in (\exists \TP) \iff \exists B \in \TP, \exists p: x \in A \iff \exists^{p(|x|)} y: \langle x, y \rangle \in B \]
	Construct an NTM that accepts A ($L(M) = A$):

	\begin{enumerate}
		\item guess y: $|y| \leq p(|x|)$
		\item run deterministic verification $\langle x, y \rangle \in B$
	\end{enumerate}
	B is in $\TP$ therefore verification is deterministic.
	As $|y|$ is bounded by $p(|x|)$ total time complexity is polynomial.

	"$\TNP \subseteq \exists \TP$"
	\[ A \in \TNP \iff \exists NTM\ M: L(M) = A \]
	Also
	\[ x \in A \iff \exists^{p(|x|)} y \]
	where $y$ encodes the accepting branch of M on input x.

	// todo why??\\
	Consequently any language $B$ of pairs $\langle x, y \rangle$ is in $\TP$.
	So, by definition
	\[ A \in \exists \TP \]

	b) using lemma \cref{quant_class_rel}
	\[ \forall \TP \xlongequal{\TP = co-\TP} \forall(co-\TP) \xlongequal{\cref{quant_class_rel}} co-\exists \TP \xlongequal{a)} co-\TNP \]

	c) trivially $\Sigma_k \subseteq (\exists \Sigma_k)$. Now reversed.

	% todo finish
	\[ A \in (\exists \Sigma_k) \iff \exists B \in \Sigma_k, \exists p: x \in A \iff \exists^{p(|x|)} y: \langle x, y \rangle \in B \]
	We want
	\[ A \in \TNP(\Sigma_{k - 1}) = \Sigma_k \]
	So, we construct an NTM M with oracle $C \in \Sigma_{k - 1}: L(M, C) = A$, we also use
	\[ B \in \TNP(\Sigma_{k - 1} \iff \exists M_B: B = L(M_B, C) \]
	algorithm of M:
	\begin{enumerate}
		\item guess y: $|y| \leq p(|x|)$
		\item run $M_B$ on $\langle x, y \rangle$.
	\end{enumerate}

	Note that
	\[ \exists y \exists z \iff \exists \langle y, z \rangle \]

	d)
	\[ \forall \Pi_k = \forall(co-\Sigma_k) \xlongequal{\cref{quant_class_rel}} co-\exists \Sigma_k \xlongequal{c)} co-\Sigma_k = \Pi_k \]

	Note that now we cannot prove e)

	f) follows from e)
	\[ \forall \Sigma_k = \forall(co-\Pi_k) \xlongequal{\cref{quant_class_rel}} co-\exists \Pi_k \xlongequal{e)} co-\Sigma_{k + 1} = \Pi_{k + 1} \]
\end{proof}
