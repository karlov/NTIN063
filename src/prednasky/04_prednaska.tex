%\section{\texorpdfstring{Polynomial Hierarchy cont}{Polynomial Hierarchy cont}}
%\vspace{5mm}
%\large

\begin{lemma}[$A^{\ast}$]
	Let $\C$ be an arbitrary class from PH.
	Then
	\[ \forall A \in \C \iff A^{\ast} \in \C \]
	Where
	\[ A^{\ast} = \{ x |\ \exists n \exists y_1 \in A \ldots \exists y_2 \in A: x = (y_1, \ldots, y_n) \} \]
	In other words, concatenation of the words from $A$ are also in $A$.

	Note that comma separators are important to stay in the same complexity class.
	We cannot guess the split by brute force.
\end{lemma}
\begin{proof} Separately for every class in PH.

	1) $C = \Sigma_0 = \Pi_0 = \Delta_0 = \TP$.\\
	$\Leftarrow$ trivially $A \subseteq A^{\ast}$ for $n = 1$.
	$x = (y)$.\\
	\[ \Rightarrow \exists DTM\ M: A = L(M) \]
	We run $M$ on all $y_1, y_2, \ldots, y_n$.
	We accept $\iff$ all computations on $y_i$ accepts.

	2) $C = \Delta_k$.
	Same as in 1). We have DTM with an oracle TM.

	3) $C = \Sigma_k, k > 0$.
	We have
	$ \exists NTM\ M$ with oracle $D \in \Sigma_{k - 1}: A = L(M, D)$.
	We simulate computation of $M$ on $y_1$.
	In every accepting path, we run $M$ on $y_2$. And so on.

	4) $C = \Pi_k$
	If there is at least one $y_i \in co-A$, whole string $(y_1, y_2, \ldots, y_n) \in (co-A)^{\ast}$.
	Therefore we proceed roughly the same as in 3), but we proceed for rejecting leaves.
\end{proof}

\begin{consequence}
	If we have $A,B,T \in \C \Rightarrow D \in \C$ where
	\[ D = \{ x |\ \exists a \in A, \exists b \in B, \exists t \in T: x = (a, b, t) \} \]

	Proof is the same, as of the lemma, but we have multiple TM.
\end{consequence}

\begin{theorem}[Polynomial hierarchy consequences]
	\[ \exists \Pi_k = \Sigma_{k + 1} \]
\end{theorem}
\begin{proof}
	\[ \exists \Pi_k \subseteq \Sigma_{k + 1} \]
	\[ A \in \exists \Pi_k \iff \exists B \in \Pi_k \exists p: x \in A \iff \exists^{p(|x|)} y: (x,y) \in B \]
	Construct an $NTM\ M$ which works as following:
	\begin{enumerate}
		\item read x
		\item guess $y: |y| \leq p(|x|)$.
		\item ask oracle if $(x,y) \in B$
	\end{enumerate}

	We get
	\[ A = L(M, B) \Rightarrow A \in \TNP(\Pi_k) = \TNP(\Sigma_k) = \Sigma_{k + 1} \]

	\[ \exists \Pi_k \supseteq \Sigma_{k + 1} \]
	Proof by induction on k.

	$k = 0$.
	\[ \Sigma_1 \subseteq \exists \Pi_0 = \exists \TP = \TNP \]

	induction hypothesis: $\Sigma_k \subseteq \exists \Pi_{k - 1}$.
	Let $A \in \Sigma_{k + 1}$ be arbitrary.
	Then
	\[ \exists NTM\ M \exists B \in \Sigma_k: A = L(M, B) \]
	$x \in A \iff \exists$ accepting computation (poly long) of $M$ on input $x$ which asks if $z_1, z_2, \ldots, z_n \in B$ and if $w_1, \ldots, w_n \in co-B$.
	Alternatively
	\[ x \in A \iff \exists^{p(|x|)} y, \exists^{p(|x|)} z = (z_1, z_2, \ldots, z_n), \exists^{p(|x|)} w = (w_1, \ldots, w_n)\]
	s.t. $y$ encodes an accepting computation of $M$ on $X$ with positive queries $z_1, \ldots, z_n$ and negative queries $w_1, \ldots, w_n$.

	We claim that the language L of pairs $(x, y) \in \TP \subseteq \Pi_k$.

	We know
	%todo rewrite
	\[ z_i \in B^{\ast} \Rightarrow z \in B^{\ast} \in \Sigma_k \subseteq^{i.h.} \exists \Pi_{k - 1}\]

	\[ w_i \in (co-B)^{\ast} \Rightarrow w \in B^{\ast} \in \Sigma_k \subseteq^{i.h.} \exists \Pi_{k - 1}\]

	So
	\[x \in A \iff \exists^{p(|x|)} y \exists^{p(|x|)} z \exists^{p_1(|z|)} y \exists^{p(|x|)} w\]
\end{proof}

\begin{definition}[PH by alternating quantifiers]\label{ph_altern}
	\[ A \in \Sigma_k \iff \exists B \in \TP \exists p: x \in A \iff \exists^{p(|x|)} y_1, \forall^{p(|x|)} y_2 \ldots : (x, y_1, y_2, \ldots, y_k) \in B \]
	Also
	\[ A \in \Pi_k \iff \exists B \in \TP \exists p: x \in A \iff \forall^{p(|x|)} y_1, \exists^{p(|x|)} y_2 \ldots : (x, y_1, y_2, \ldots, y_k) \in B \]
\end{definition}
\begin{proof}
	Proof by induction on $k$.

	For $k = 0$ we have no quantifiers.
	\[ A \in \TP \exists B \in \TP \]
	we can take $A = B$.

	$k \to k + 1$. Let $A \in \Sigma_{k + 1} = \exists \Pi_k$.
	Then
	\[ \exists \Pi_k \exists p: x \in A \iff \exists^{p(|x|)} y: (x,y) \in B \]
	By the i.p.
	% todo finish, 4th lecture
	\[ \Rightarrow (x,y) \in B \iff \forall^{p(|x|)} y_1, \exists^{p(|x|)} y_2 \ldots : ((x,y), y_1, y_2, \ldots, y_k) \]
\end{proof}

\begin{example}
	Language from $\Sigma_2$.
	Optimization version (Boolean minimization).

	Input: CNF $F$\\
	Output: CNF $H: H \equiv F \land |H|$ is minimal (we can define minimal in many ways, e.g. minimal in bits to represent).

	Now we convert into decision problem.\\
	Input: CNF $F, k \in \N$ \\
	Question: $\exists CNF\ H: F \equiv H \land |H| \leq k $.

	Problem is in $\Sigma_k$ since
	\[ F \in BM \iff \exists^{|H| \leq |F|} H, \forall (x_1, \ldots, x_n): H \leq k \land F(x_1, \ldots, x_n) = H(x_1, \ldots, x_n) \]

	We also know $BM \in \Sigma_2-complete$.
\end{example}

\begin{note}
	If we have a hard problem, it can be encoded in CNF and solved by CNF machinery.

	CNF solvers are highly optimized over more than 20 years.
\end{note}
