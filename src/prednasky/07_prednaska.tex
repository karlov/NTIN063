\section{\texorpdfstring{TM with advice}{TM with advice}}
\vspace{5mm}
\large
% 8th lecture

\begin{definition}[$DT(T(n))|_{a(n)}$]
	Let $T, s: \N \to \N$. Then
	\[ DT(T(n))|_{a(n)} \]
	is a class of languages recognizable by DTM in time $T(n)$ with $s(n)$ bit advice $\alpha_n$.
	\[ L \in DT(T(n))|_{s(n)} \iff \exists DTM M \& \exists \{\alpha_n\}_{n \in \N}, \alpha_n \in \{ 0, 1 \}^{s(n)}: \]
	\[ \forall x \in \{ 0, 1 \}^n: x \in L \iff M(x, \alpha_n) = 1 \]
	And M makes $\bigO(T(n))$ steps on the input $(x, \alpha_n)$.
\end{definition}

\begin{note}
	TM with advice is an non-uniform computation as advice size depends on input size.
\end{note}

\begin{theorem}[$\Ppoly$]
	$\Ppoly = \bigcup_{c, d} DT(n^c)|_{n^d} $
\end{theorem}
\begin{proof}
	"$\subseteq$".
	Let $L \in \Ppoly$ then by definition exists polynomial time circuit $\{ C_n \}$ that recognizes $L$.
	We set $\{ \alpha(n) \} = \{ C_n \}$ (via encoding) and the DTM M simulates $C_n$ on $x$.

	As circuit is of polynomial size, M works in polynomial time.
	Therefore
	\[ L \in \bigcup_{c, d} DT(n^c)|_{n^d} \]

	"$\supseteq$". Let $L \in \bigcup_{c, d} DT(n^c)|_{n^d} \Rightarrow \exists DTM M$ which works in time $T(n)$, uses advice $\{ \alpha(n) \}$ of size $a(n)$ and $L = M(x, \alpha_n)$.

	Like in proof $\TP \subseteq \Ppoly$ \cref{tp_ppoly} there exists a polynomial size family of $\{ C_n \}$:
	\[ \forall x \in \{ 0, 1 \}^n \forall \alpha \in \{ 0, 1 \}^{a(n)}: M(x, \alpha) = C_m(x, \alpha), m = n + a(n) \]

	Now we hardcode $\{ \alpha_n \}$ into the circuit.
	Define
	\[ C_m^{hard}(x) = C_m(x, \alpha_n) \]
	Then
	\[ x \in L \iff M(x, \alpha_n) = 1 \iff C_m(x, \alpha_n) = 1 \iff C_m^{hard}(x) = 1 \Rightarrow L \in \Ppoly \]
\end{proof}

\begin{note}
	Not every boolean function can be computed by a polynomial size circuit.
\end{note}

\begin{theorem}[$\exists f$ no poly circuit]
	\[ \forall n > 1, \exists f: \szo^n \to \szo \]
	such that $f$ cannot be computed by circuit $|C| = \frac{2^n}{10n}$.
\end{theorem}
\begin{proof}
	Idea: show the mismatch between \# of functions and \# of circuits.

	Let $C$ be a circuit, $|C| = m$.
	Since every elementary gate has at most 2 inputs, $C$ has $\leq 2m$ edges.
	Therefore $C$ can be represented by a binary string of length $3m \log m$.
	So
	\[ |\{ C |\ |C| = m \}| = 2^{3m \log m} \]
	However, there are $2^{2^n}$ functions.

	Setting $m = \frac{2^n}{10n}$ we have at most
	\[ 2^{3 \frac{2^n}{10n} \log (\frac{2^n}{10n})} \leq 2^{3 \frac{2^n}{10n} n} = 2^{\frac{3}{10} 2^n} < 2^{2^n} \]
	circuits.
\end{proof}

\begin{consequence}
	With growing $n$ more and more functions are not computable by polynomial circuits.
\end{consequence}

\begin{note}[Open Problem]
	If we can find a family of functions $\{ f_n\}$ that are not computable by polynomial size circuit such that $L \in \TNP$.
	Would imply $\TNP \nsubseteq \Ppoly \Rightarrow \TP \neq \TNP$.
\end{note}

\begin{definition}[$\Pi_n, \Sigma_n$-SAT]
	\[ \Sigma_n-SAT = \exists x_1 \forall x_2 \ldots Q x_n \varphi(x_1, \ldots, x_n) = 1 \]
	\[ \Pi_n-SAT = \forall x_1 \exists x_2 \ldots Q x_n \varphi(x_1, \ldots, x_n) = 1 \]
	Where $Q$ is either $\exists$ of $\forall$ depending on parity of $n$, $\varphi \in CNF$.
\end{definition}

\begin{lemma}[$\Pi_n, \Sigma_n$-complete language]
	$\Sigma_n-SAT$ is $\Sigma_n$-complete.
	$\Pi_n-SAT$ is $\Pi_n$-complete.
\end{lemma}
\begin{proof}
	For $\Sigma_n$ only as proof for $\Pi_n$ is the same.

	Let $L \in \Sigma_n$ arbitrary.
	By the PH definition via alternating quantifiers \cref{ph_altern}:
	\[ \exists p \exists L_{po} \in \TP: x \in L \iff \exists^{p(n)} y_1, \forall^{p(n)} \ldots Q^{p(n)} y_n: (x, y_1, \ldots, y_n) \in L_{po} \]
	Let $M$ be a DTM which accepts $L_{po}$.
	From Cook-Levin theorem we can encode the computation of $M$ by a CNF $\varphi_n$:
	\[ x \in L \iff \exists^{q(|x|)} t_1, \forall^{q(|x|)} \ldots Q^{q(|x|)} t_n: \varphi_n(v, t_1, \ldots, t_n) = 1 \]
	where $v$ encodes input $x$ and $t_i$ encodes $y_i$ (both as binary string).
	All encodings are of size $q(|x|)$.

	$L$ can be a language over any alphabet, however the blowup of binary encoding would be logarithmic with respect to alphabet size.
\end{proof}

\begin{theorem}[Karp-Lipton(1980)]
	$\TNP \subseteq \Ppoly \Rightarrow PH = \Sigma_2^p$.
\end{theorem}
\begin{proof}
	% todo check
	Suffices to prove $\Pi_2 \subseteq \Sigma_2$ rest follows from \cref{ph_collapse}???
	To do that we need to prove
	\[\exists L \in \Pi_2\text{-complete}\ \& L \in \Sigma_2 \]

	Idea: show $\Pi_2$-SAT $\in \Sigma_2$.

	$\Pi_2$-SAT is a language of following CNF formulas:
	\begin{equation}
		\forall x \in \szo^n \exists y \in \szo^n: \varphi(x, y) = 1
		\tag{$*$}
	\end{equation}
	If we fix $x$ we get a language
	\[ L_x = \{ (\varphi, x) |\ \exists y \in \szo^n: \varphi(x, y) = 1 \} \]
	Then
	\[ \forall x: L_x \in \Sigma_1 = \TNP \stackrel{\text{assumption}}{\subseteq} \Ppoly \]
	Therefore $\exists \{ C_k \}: |C_k| = p(k)$ such that
	\[ \forall \varphi_{n + m} \forall x \in \szo^n: C_k(\varphi_{n + m}, x) = 1 \iff \forall y \in \szo^n \varphi(x, y) = 1 \]
	Where $k$ is \# of bits to encode $(\varphi, x)$.

	Main idea: change $\{ C_k \}$ to a new family $\{ C_k^{out} \}$ which also outputs the satisfiable assignment.
	Meaning if $C_k$ outputs 1 then $C_k^{out}$ outputs $y \in \szo^n: \varphi(x, y) = 1$.

	$C_k^{out}$ outputs such $y$ bit by bit by chaining circuits that do $\varphi \land y[i]$ where $i$ is the bit position in $y$.
	We assume $|C_k^{out}| \leq q(n)$.
	We can encode $C_k^{out}$ into a bit string of length $r(k) \leq \bigO(q^2(n))$.

	Then,
	\begin{equation}
		\exists y \in \szo^{r(k)} \forall x \in \szo^n: \varphi(x, y) = C_k^{out}(x, (\varphi, x)) = 1
	\tag{$**$}
	\end{equation}

	Claim $(*) \iff (**)$ since if $(*)$ is true than $\exists$ circuit that on input $(\varphi, x)$ outputs $y$ such that $\varphi(x, y) = 1$.
	If $(*)$ is false $\Rightarrow \exists x: \forall y: \varphi(x, y) = 0$ then $(**)$ is also false.

	Also $(**)$ is a $\Sigma_2$ formula, therefore $\Pi_2$-SAT $\in \Sigma_2$.

\end{proof}

\subsection{Classes AC, NC}
% 9th lecture

\begin{definition}[Class NC]
	$L \in NC^i$ if there exists polynomial size family of circuits $\{ C_n \}_{n \in \N}$ accepting $L$ such that $\forall n$ depth of $C_n \in \bigO(\log^i n)$.

	\[ NC = \bigcup NC^i \]
\end{definition}

\begin{definition}[Class AC]
	$AC^i$ is defined similarly to $NC^i$ but gates may have arbitrary \# of input and output edges.
\end{definition}

\begin{theorem}[NC and AC]\label{ac_dc}
	\[ \forall i: NC^i \subseteq AC^i \subseteq NC^{i + 1} \]
\end{theorem}
\begin{proof}
	By definition:
	\[ \forall i: NC^i \subseteq AC^i \]

	Let $L \in AC^i$ arbitrary, then there is a circuit $\{ C_n \}_{n \in \N}$ with depth $\bigO(\log^i n)$ that recognizes $L$.
	$\{ C_n \}_{n \in \N}$ is a polynomial size family of circuits $\Rightarrow \exists p: |C_n| \leq p(n)$.
	Then, every gate in $C_n$ has fan-in $\leq p(n)$.
	Replace every gate in $C_n$ by a binary tree of fanin-2 gates with depth $\log p(n) \in \bigO(\log n)$.
	% todo picture lecture 9 12:00
	Therefore, we can construct $C_n^{new}$ with fanin-2 of depth $\bigO(\log^{i + 1} n)$ that simulates $C_n \Rightarrow L \in NC^{i + 1}$.
\end{proof}

% book chapter 6.5.1
\subsection{Connection between NC and parallel computing}

\begin{definition}[Parallel computer]
	Has $n$ processors, communication between them with global clock.
	Maximum "distance" (shortest path in graph of processors) between processors is $\bigO(\log n)$.

	For example hypercube architecture: labels of processors are binary numbers with $\log n$ bits.
	Processors are connected if their labels differ by 1 bit, Hamming distance is 1.
\end{definition}

\begin{definition}[Efficient parallel algorithm]
	A computational task is said to have \emph{efficient parallel algorithms} if inputs of size $n$ can be solved using a parallel computer with $n^{\bigO(1)}$ processors and in time $\log^{\bigO(1)} n$.

	Source ~\cite[p.~109]{arora2009computational}.
\end{definition}

\begin{example}[Transitive closure]
	Problem of finding a transitive closure of directed graph can be solved by efficient parallel algorithm.
	Transitive closure adds an $(a, b)$ edge for every $Path(a, b)$.

	Instead of running multiple instances of BFS in parallel, we can multiple the adjacency matrix of a graph.
	The reachability matrix can be constructed in 1 multiplication $\bigO(1)$ using $n^3$ processors in time $\bigO(\log n)$.
\end{example}

Q: Do we assume that every processor outputs 1 bit?\\
A: No, each processor is a TM with output tape. As each of them works in $\log$ time, is uses at most $\log$ space which yields a polynomial size output.

\begin{theorem}[NC and parallel computing]
	$L$ has an efficient parallel algorithm (which recognizes $L$) $\iff L \in NC$.
\end{theorem}
\begin{proof}
	"$\Rightarrow$". Let efficient parallel algorithm for $L$ has $n^c$ processors and works in time $\bigO(\log^d n)$.
	Design a circuit $\{ C_n \}$ with $\bigO(n^c \cdot \log^d n)$ processors.

	"Gates" we construct (more precicely, fixed size circuits) are organized in $\bigO(\log^d n)$ levels.
	Gate at $j$-th level simulates processor in $j$-th step of the computation.
	The input edges comes from the previous levels and corresponds to the communication channels between the processors.

	Then $L \in AC^d$, by \cref{ac_dc} $L \in NC^{d + 1} \Rightarrow L \in NC$.

	"$\Leftarrow$" we have a polynomial size circuit with $\log$ depth.
	Replace every gate by a processor and interconnect them in some standard way.
	Sumilating $C_n$ getting all signals from level $j$ to $(j + 1)$ may take $\bigO(\log n)$ steps of the parallel computing.
	Therefore simulation takes $\bigO(\log^{d + 1} n)$.
\end{proof}

Q: what is the relationship between $NC$ and $\Ppoly$?\\
A: $NC, AC \subseteq \Ppoly$, these classes allow fast computaton. Similar to linear or quadratic subclasses in $\TP$.
