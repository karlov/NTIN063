\section{\texorpdfstring{Ladner theorem}{Ladner theorem}}
\vspace{5mm}
\large

% 11th lecture
\begin{theorem}[Ladner]
	If $\TP \ne \TNP \Rightarrow \exists L \in \TNP \setminus \TP \& L \notin \TNP$-complete.

	Proves existence of $\TNP$-intermediate problems.
	Candidates for the $\TNP$-intermediate problems are \emph{Factoring} or \emph{Graph isomorphism}.
\end{theorem}
\begin{proof}
	Idea: find a language $L \in \TNP \cap co-\TNP \land L \notin P$.
	Language we construct is artificial.

	Not all technical details of the proof will be presented.

	Redefine SAT as
	\[ SAT = \{ i \in \szo^{\ast} |\ \text{CNF formula with code i is satisfiable} \} \in \TNP \]
	Let $M_1, M_2, \ldots$ be a list of oracle DTM using $\szo$ alphabet and running in polynomial time.

	Note:
	\begin{gather*}
		x \in \TP \Rightarrow \exists i: X = L(M_i, \emptyset) \\
		x \in \TNP-c \Rightarrow \exists i: SAT = L(M_i, X)
	\end{gather*}
	Our goal is to construct language $A$ such that:
	\begin{gather*}
		A \notin \{ L(M_1, \emptyset), L(M_2, \emptyset), \ldots \} \\
		SAT \notin \{ L(M_1, A), L(M_2, A), \ldots \} \\
	\end{gather*}

	$A$ is defined as $L(N)$ where $N$ is an universal NTM that works in time $p(n)$ and uses subroutines:
	\paragraph{Agree($M_i, \emptyset, y$)}
	\begin{enumerate}
		\item $N$ simulates $M_i$ with oracle $\emptyset$ on $y$ and outputs YES if $M_i$ accepts.
			Outputs NO if $M_i$ rejects.
		\item $N$ checks if $y \in A$. //here A simulates itself
		\item if 1) and 2) have the same result - output YES, o/w NO.
	\end{enumerate}

	\paragraph{Agree-O($M_i, A, y$)}
	\begin{enumerate}
		\item $N$ simulates $M_i$ with oracle $A$ on $y$ and outputs YES if $M_i$ accepts.
			Outputs NO if $M_i$ rejects.

		//here A simulates itself
		\item $N$ checks if $y \in SAT$.
		\item if 1) and 2) have the same result - output YES, o/w NO.
	\end{enumerate}
	$N$ works on input $x \in \szo$ with following algorithm ($i$ is the Godel number of TM, $y$ is current string)
	\begin{enumerate}
		\item $i = 1, y = ""$ // empty string
		\item until($N$ makes $p(|x|)$ steps) repeat
		\item \tab while(AGREE($M_i, \emptyset, y$)) do $y = $ next($y$) // increment
		\item \tab while(AGREE-O($M_i, A, y$)) do $y = $ next($y$) // increment
		\item \tab $++i$ // get next TM
		\item if main loop terminates in step 3 then $N$ accepts $\iff x \in SAT$.
		\item else $N$ rejects $x$.
	\end{enumerate}

	$A \in \TNP$ as it is accepted by NTM in polynomial time.

	\paragraph{$A \notin \TNP$-complete} Assume by contradiction that $A \in \TNP$-complete.
	Then $\exists k: SAT = L(M_k, A)$ also let $k$ be smallest possible.
	Let $j < k$ be arbitrary and let $x$ be large enough.
	As by assumption $A \in \TNP$-complete $\Rightarrow A \notin \TP$ as it would contradict $\TP \ne \TNP$.
	Therefore $A \ne L(M_j, \emptyset) \Rightarrow \exists y: AGREE(M_j, \emptyset, y)$ outputs NO.
	Then main loop \emph{breaks} from "while" in step 3, because $N$ will eventually find such $y$.

	% todo why is it different from SAT?
	Moreover, main loop \emph{breaks} from "while" in step 4 for some $l \ne k$ as $SAT \ne L(M_l, A) \Rightarrow \exists y: AGREE-O(M_l, A, y)$ outputs NO.
	Finally, main loop never leaves step 4 for $k$.

	Altogether, $\exists c, \forall x: |x| \geq c$ the main loop terminates in step 4 and rejects $x$.
	So, $N$ accepts a finite language $\Rightarrow A \in P$ which is a contradiction.

	\paragraph{$A \notin \TP$} Assume by contradiction that $A \in \TP$.
	Then $\exists k: A = L(M_k, \emptyset)$ also let $k$ be smallest possible.
	Let $x$ be large enough.

	For $j < k$ main loop \emph{breaks} from "while" in step 3 as we assume $k$ is the smallest.
	Also, main loop \emph{breaks} from "while" in step 4 as we assume $k$ is the smallest.
	As $A \in \TP \& SAT \ne L(M_j, A)$.
	Finally, main loop never leaves step 3 for $k$.
	Therefore, $N$ can accept $x$ only in step 2 of Agree procedure.
	So
	\[ \exists c, \forall x: |x| \geq c: x \in A \iff x \in SAT \Rightarrow A \in NPC \]
	Which is a contradiction with assumption $A \in \TP$.

	\paragraph{Technical details}
	Note that hardest technical detail is the way $N$ simulates $A$ (itself).
	Q: again lier's paradox (self-reference)?

	Also, $N$ has to check 2 conditions in both subroutines.
	Therefore, if the branch is accepting, $N$ returns to the branching point and compares the results.
	However, such approach does not work for rejecting branches.
	One of the solution is to choose $|y| \leq \log |x| \Rightarrow 2^{|y|} \leq |x|$ and $N$ can check all branches.
\end{proof}
