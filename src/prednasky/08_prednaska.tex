\section{\texorpdfstring{Randomized computation}{Randomized computation}}
\vspace{5mm}
\large

% 9th lecture from 51:50

\begin{definition}[Probabilistic TM (PTM)]
	TM with 2 transition functions: $\delta_1, \delta_2$.
	When PTM works on $x$ it at each step selects $\delta_1$ with probability $\frac{1}{2}$ (same for $\delta_2$).
	PTM uses a fair cons toss that is independent from the previous results.

	PTM M returns a random variable denoted $M(x)$, $M(x) = 1$ if $M$ accepts $x$ and $0$ o/w.

	We say that $M$ runs in tim $T(n)$ if for every input $x$ M stops after at most $T(|x|)$ steps, regardless of the selections.
\end{definition}

\begin{notation}
	For a language $L \in \szo^{\ast}$ and $x \in \szo^{\ast}$ we define:
	\[ L(x) = 1 \iff X \in L \& L(x) = 0 \iff x \notin L \]
\end{notation}

\begin{definition}[BPP]
	PTM M accepts the language $L \in \szo^{\ast}$ in time $T(n)$ if
	\[ \forall x \in \szo^{\ast}: Pr[M(x) = L(x)] \geq \frac{2}{3} \]
	And $M$ runs in $T(n)$.

	$BPTIME(T(n))$ is a class of languages accepted by PTM in time $T(n)$.
	\[ BPP = \bigcup BPTIME(n^c) \]
	BPP - bounded error probabilistic polynomial time.
\end{definition}

\begin{properties}[BPP]
	BPP is invariant to (robust):
	\begin{itemize}
		\item Accepting probability: BPP will remain the same if we replace $\frac{2}{3}$ by any fraction $\in (\frac{1}{2}, 1)$.

		\item Probabilities of picking $\delta_1$ or $\delta_2$ can also be changed to $p, 1 - p$ instead of $\frac{1}{2}$.
		\item Running time: $T(n)$ can be replaced by "expected" $T(n)$.
	\end{itemize}
\end{properties}

\begin{note}[Open Question]
	We know:
	\[ \TP \subseteq BPP \]
	as we can take $\delta_1 = \delta_2$.

	However the strict inclusion is an open question
	\[ \TP \subsetneq BPP \]
\end{note}

\begin{note}[BPP and NP]
	The idea is similar, but BPP requires $\frac{2}{3}$ of accepting computation branches and $\TNP$ only 1.
\end{note}

\begin{definition}[Probabilistic TM (PTM) Alternative]
	Randomness can be presented making PTM accept additional input string, which is random $y \in \szo^{T(n)}$.
	Each bit tells which transition function to choose.
\end{definition}

% todo change to Fact and fix cref
\begin{theorem}[Chernoff]\label{chernoff}
	Let $x_i, \ldots, x_n$ be independent r.v. such that $x_i \in \szo$ and let $\mu = \sum \E[x_i]$ then
	\[ \forall c > 0: \Prob[|\sum x_i - \mu| \geq c \cdot \mu ] \leq 2e^{-\min\{ \frac{c^2}{4}, \frac{c}{2} \} \cdot \mu } \]
\end{theorem}

\begin{theorem}[BPP error reduction]\label{error_reduction}
	Let $L \in \szo^{\ast}$ be a language, $c > 0$ constant.
	Assume that there $\exists PTM\ M$ such that
	\[ \forall x \in \szo^{\ast}: Pr[M(x) = L(x)] \geq \frac{1}{2} + |x|^{-c} \]
	Note that as $x$ grows, error $|x|^{-c} \to 0$.

	Then
	\[ \forall d > 0: \exists PTM \ M_e: \forall x \in \szo^{\ast}:Pr[M_e(x) = L(x)] \geq 1 - 2^{-|x|^d} \]

	And if $M$ work in polynomial time so does $M_e$.
\end{theorem}
\begin{proof}
	Idea: $M_e$ runs $M$ $k = 8 \cdot |x|^{2d + c} + 1$ times.
	Assume $k$ is odd.
	Obtaining results $y_1, y_2, \ldots y_k$ and $M_e$ accepts iff
	\[ \sum y_i > \frac{1}{2} k \]
	Majority decision.

	Denote $x_i \in \szo^{\ast}$ r.v. where
	\[ x_i = 1 \iff y_i = L(x) \]
	As each $y_i$ represents different branch of computation, and coin flip is independent, $x_i$ are independent.
	Then by assumption on $M$
	\[ \E[x_i] = \Prob[x_i = 1] \geq \frac{1}{2} + |x|^{-c} = p \]

	Define $\delta = \frac{1}{2} |x|^{-c}$ and consider
	\[ S = \sum x_i \geq pk - \delta pk = pk (1 - \delta) = k \left(\frac{1}{2} |x|^{-c}\right)\left(1 - \frac{1}{2} |x|^{-c}\right) = \frac{1}{2}k + k\left(\frac{3}{4} |x|^{-c} - \frac{1}{2} |x|^{-2c}\right) \]
	Then as $\left(\frac{3}{4} |x|^{-c} - \frac{1}{2} |x|^{-2c}\right) \geq 0$
	\[ S = \frac{1}{2}k + k\left(\frac{3}{4} |x|^{-c} - \frac{1}{2} |x|^{-2c}\right)  > \frac{1}{2} k \]
	We conclude that if $M$ answers correctly at least $> \frac{1}{2} k$ then $M_e$ answers correctly (for sure).

	On the contrary, $M_e$ can make a mistake only if
	\[ \sum x_i < pk - \delta pk \]
	So
	\[ \Prob[\sum x_i < pk - \delta pk] \]
	is an upper bound of the probability that $M_e$ makes a mistake.
	Then
	\[ \E[x_i] \geq p \Rightarrow \sum \E[x_i] \geq pk \]
	For some $\varepsilon > 0$ denote
	\[\mu = \sum \E[x_i] = pk + \varepsilon \]
	Finally compute
	\[ P_e = \Prob[ pk - \sum x_i > \delta pk] \leq \Prob[ pk - \sum x_i + (1 - \delta)\varepsilon > \delta pk] = \Prob[pk + \varepsilon - \sum x_i > \delta pk + \delta \varepsilon] = \]
	\[ = \Prob[\mu - \sum x_i > \delta \mu ] = \]
	As the probability is symmetric and by Chernoff \cref{chernoff}
	% todo picture 9th lecture 01:58:50
	\[ = \frac{1}{2} \Prob[|\mu - \sum x_i| > \delta \mu ] \leq \frac{1}{2} 2 e^{-\frac{\delta^2}{4} \mu} \leq e^{-\frac{\delta^2}{4} pk} \]
	To summarize:
	\[ P_e = \Prob[M_e\ \text{makes an error}] \leq e^{-\frac{1}{4}\frac{|x|^{-2c}}{4} (\frac{1}{2} + |x|^{-c}) 8 \cdot |x|^{2d + c} + 1} = e^{-\frac{8 \cdot |x|^{2c + d}}{32 \cdot |x|^{2c}} + \frac{8 \cdot |x|^{2c + d}}{16 \cdot |x|^{3c}}} \]
	By reducing the exponent (2nd summand) and simplifying the first
	\[ P_e \leq e^{-\frac{1}{4} |x|^d} = (e^{\frac{1}{4}})^{-|x|^d} \leq 2^{-|x|^d} \]
\end{proof}

\begin{lemma}[Fair coin]
	A fair coin can be simulated by a PTM with an access to a biased coin with $\Prob[\text{head}]=p\in(0,1)$ in $\bigO(1)$ expected time.
\end{lemma}
\begin{proof}
	In each round PTM flips the biased coin twice if
	% todo readability
	\[ \twopartdef{H + H}{}{T + T}{} \Rightarrow \text{new pair of flips} \]
	Otherwise
	\[ \twopartdef{H + T}{H, \Prob[H] = p (1 - p)}{T + H}{T, \Prob[H] = (1 - p) p}\]
	As
	\[ \Prob[\text{PTM stops at round}\ i] = 2 p (1 - p) = c, c \in (0, 1) \]
	Then expected \# of trials before PTM stops
	\[ \sum i (1 - c)^{i - 1} c \]
	which converges.
\end{proof}
\begin{lemma}[Biased coin]
	A biased coin with probability $\Prob[\text{head}]=\rho$ can be simulated in $\bigO(1)$ time by a ``standard'' PTM (with fair coin) if the $i$-th bit of $\rho$ can be generated in $\text{poly}(n) = n^c$.
\end{lemma}
\begin{proof}
	Let $\rho$ be a binary written as
	\[ \rho = 0p_1p_2 \ldots \]
	PTM generates fair sequence $b_1, b_2, \ldots$.
	If PTM in $i$-th step generates $b_i$ then
	\begin{itemize}
		\item $b_i < p_i \Rightarrow H$
		\item $b_i > p_i \Rightarrow T$
		\item $b_i = p_i$ go to step $(i + 1)$.
	\end{itemize}
	M gets to $i$-th step $\iff$
	\[ \forall 1 \leq j \leq i - 1: b_j = p_j \]
	which occurs with probability $\frac{1}{2^{i - 1}}$.
	% todo did not understand following implication
	$p_i = 1 \Rightarrow$ return $H$ in step $i$ with probability $\frac{1}{2}$.\\
	$p_i = 0 \Rightarrow$ return $H$ in step $i$ with probability $0$.
	Then
	\[ \Prob[H] = \sum p_i \frac{1}{2^i} = \rho \]

	The expected time $\sum n^i \frac{1}{2^{i - 1}}$ which converges.
\end{proof}

\begin{definition}[Expected running time]
    Let $M$ be a PTM and $x$ an input.
    Then $T_{M,x}$ is a random variable which denotes the time it takes $M$ to halt on $x$.
    The PTM $M$ runs in expected time $T(n)$ if $\forall x \in \szo^{\ast}: \E[T_{N,x}] \leq T(|x|)$.
\end{definition}

\begin{theorem}[Markov inequality]\label{markov}
	Let X be a non-negative r.v. Then
	\[ \Prob[ X \geq k \E[x]] \leq \frac{1}{k} \]
\end{theorem}

\begin{observation}
	Let PTM $M$ have expected time complexity $T(n)$.
	If we simulate $M$ on PTM $M_s$ with time complexity $100 T(n)$ (by rejecting longer computations), then the probability (by Markov \cref{markov})
	\[ \Prob[M\ \text{runs longer than}\ 100 T(n)] \leq \frac{1}{100} \]
	Therefore, $M_s$ rejects if it stops because of the clock, o/w it does the same as $M$.
	Then $\Prob[M_s\ \text{makes an error}]$ is at most $100$ greater than for $M$.

	We can reduce the error back to the error of $M$ by \cref{error_reduction}.
\end{observation}

% 10th lecture
\subsection{Classes RP, ZPP}

\begin{definition}[RTIME, RP]
	$L\in\text{RTIME}(T(n))\Leftrightarrow\exists$ PTM $M$ working in (expected) time $T(n)$ such that
    \begin{gather*}
    	x\in L\Rightarrow \Prob[M(x)=1]\geq 2/3\\
	x\not\in L\Rightarrow\Prob[M(x)=0]=1
    \end{gather*}

    $RP=\bigcup_c\text{RTIME}(n^c)$ -- Randomized Polynomial time

    Note that we can again prove that class is the same with either expected or exact polynomial time.
\end{definition}
\begin{definition}[ZTIME, ZPP]
    $L\in\text{ZTIME}(T(n))\Leftrightarrow\exists$ PTM $M$ working in (expected) time $T(n)$ such that
    \begin{gather*}
    	x\in L\Rightarrow \Prob[M(x)=1]=1\\
	x\not\in L\Rightarrow\Prob[M(x)=0]=1
    \end{gather*}

    $ZPP=\bigcup_c\text{ZTIME}(n^c)$ -- Zero-error Probabilistic Polynomial time

    Note that we can again prove that class is the same with either expected or exact polynomial time.
\end{definition}
\begin{lemma}[On probabilistic classes]
    \begin{enumerate}
        \item $RP\subseteq \TNP$
        \item $BPP=co-BPP$
        \item $RP\subseteq BPP, co-RP\subseteq BPP$
        \item $ZPP=RP\cap co-RP$
    \end{enumerate}
\end{lemma}
\begin{proof}
    \begin{enumerate}
	    \item
		For $\TNP$ we require at least one accepting computation, by definition of $RP$ we have at least $\frac{2}{3}$ of them.
		If we allow PTM to work in expected polynomial time, then there is shorter accepting branch of computation.
        \item Design new PTM which simulates original PTM but flips all answers.
		\[ \Prob[\overline{M}(x) = \overline{L}(x)] = \Prob[1 - M(x) = 1 - L(x)] = \Prob[M(x) = L(x)] \geq \frac{2}{3} \]
	\item Follows from stricter definition of $RP$. Second condition follows from 2).
        \item $ZPP=RP\cap co-RP$
    \end{enumerate}
\end{proof}
\begin{theorem}[$BPP \subseteq \Ppoly$]
	\[ BPP \subseteq \Ppoly \]
\end{theorem}
\begin{proof}
    TODO
\end{proof}
\begin{theorem}[Sipser-Gács-Lautemann]
    BPP$\subseteq \Sigma^p_2 \cap \Pi_2^p$
\end{theorem}
\begin{proof}
    TODO
\end{proof}
