\section{\texorpdfstring{PCP theorem}{PCP theorem}}
\vspace{5mm}
\large

% 12th lecture
Many problems in $\TNP$ are artificial versions of optimization problems.

\begin{example}[Knapsack]
	Having $n$ items with sizes $s_1, \ldots, s_n$ and prices $p_1, \ldots, p_n$.
	Also capacity $c$ - size of knapsack.

	Optimization problem: minimize $\sum p_i$ subject to constraint $\sum s_i \leq c$.
	Optimization knapsack is $\TNP$-hard.

	Decision problem: add a parameter $k$.
	Input: $(s_1, \ldots, s_n, p_1, \ldots, p_n, c, k)$. \\
	Question: $\exists A \subseteq [n]: \sum_{i \in A} p_i \geq k \& \sum_{i \in A} s_i \leq c$.

	Decision knapsack is $\TNP$-complete.
\end{example}

\begin{example}[Subset sum]
	Is a modification of Knapsack.\\
	Input: $a_1, \ldots, a_n, b$.\\
	Question: $\exists A \subseteq [n]: \sum_{i \in A} a_i = b$.

	Reduction from Knapsack: set $s_i = a_i \& c_i = a_i$.
\end{example}

Despite Knapsack being $\TNP$-hard, there is an approximation algorithm.

\begin{definition}[FPTAS]
	Fully polynomial time approximation scheme.
	Algorithm $A$ with 2 inputs:
	\begin{enumerate}
		\item size of the problem $x$.
		\item allowed precision (relative error) $\varepsilon$.
	\end{enumerate}
	Such that $\frac{|A(x) - OPT(x)|}{OPT(x)} \leq \varepsilon$.
	$A$ runs in polynomial time with respect to $|x|$ and $\frac{1}{\varepsilon}$.
\end{definition}

Knapsack has a FPTAS algorithm, therefore it can be efficiently approximated with arbitrary error.
On the other hand, Traveling salesman problem (TSP) does not have such algorithm.

\begin{example}[TSP]
	Input: complete undirected graph $G = (V, E)$ and function $l:E \to \N$. \\
	Output: shortest Hamiltonian circuit in $G$.
\end{example}

\begin{lemma}[Inaproximation of TSP]
	If there exists polynomial time approximation algorithm $A$ for TSP with ratial error $\rho = \frac{A(x)}{OPT(x)}$.
	Where $\rho \geq 1$ is constant.
	Then $\TP = \TNP$.
\end{lemma}
\begin{proof}
	Let us take an instance of HC problem: graph $G = (V, E), |V| = n$, question: $\exists HC \in G$?
	And convert it into TSP problem
	\[ G_t = (V, E = V \times V) \]
	and the lengths:
	\[ l(e) = \twopartdef {1}{ e \in E} { \rho n + 1 }{e \notin E\ \text{artificial edge}} \]

	If the initial problem is answered YES then $OPT(x) = n$ and the algorithm $A$ for TSP outputs a HC of length $\leq n\rho$.
	Therefore, $A$ must output HC which was in the initial graph.

	On the contrary, if the initial answer is NO, then $A$ outputs $OPT(x) > n \rho$.
	GAP construction no HC in $(n, n \rho]$.

	Therefore a polynomial approximation algorithm for TSP gives a polynomial algorithm for HC.
\end{proof}

\begin{definition}[MAX-3-SAT]
	For a 3-CNF formula $\varphi, val(\varphi) = \max$ fraction of clauses that can be satisfied by an assignment.

	Note: $\varphi$ is satisfiable $\iff val(\varphi) = 1$.

	For $\rho \leq 1$ and algorithm $A$ is $\rho$-approximation algorithm in MAX-3-SAT if
	$ \forall \varphi $ with $m$ clauses $A(\varphi)$ outputs and assignment satisfying $\geq \rho \cdot val(\varphi) \cdot m$ clauses.

	In other words $OPT(\varphi) = n$ and
	\[ \frac{A(\varphi)}{OPT(\varphi)} \geq \rho \cdot m \]
\end{definition}

\begin{exercise}[MAX-3-SAT algorithm]
	Greedy algorithm for MAX-3-SAT with $\rho = \frac{1}{2}$:
	\begin{enumerate}
		\item take variable $x_i$
		\item count all clauses where $x_i$ or $\neg x_i$ is present.
			Take majority and assign $x_i = 1, x_i = 0$ (resp) and remove opposite.
	\end{enumerate}
	At every step we satisfy at least half of clauses that contains $x_i$.

	What is the expected \# of satisfied clauses for a random assignment?
	Assume, every clause has 3 distinct variables, therefore is has 8 possible assignments and only 1 of them does not satisfy the clause.
	Therefore, $\E[rng] \geq \frac{7}{8} \cdot m$ yielding $\rho = \frac{7}{8}$.
	Also, $\forall \varepsilon > 0$ there is a deterministic algorithm with $\rho = \frac{7}{8} - \varepsilon$ (derandomization).

 \end{exercise}

 \begin{note}
	PCP theorem implies that if there is an approximation algorithm with $\rho = \frac{7}{8} + \varepsilon \Rightarrow \TP = \TNP$.
 \end{note}

\begin{exercise}[Vertex cover algorithm]
	Input: undirected graph $G = (V, E)$\\
	Output: minimal vertex cover:
	\[ S \subseteq V: \forall e = (a, b) \in E: \{ a, b \} \cap S \ne \emptyset \]
	Even though, Vertex cover is the dual problem for Independent set, the approximation approaches are different.

	% todo finish lecture 12 from 01:02:00
 \end{exercise}

 \begin{reminder}[$\TNP$ verifier]
	 Verifier for $L \in \TNP$ is a polynomial time DTM:
	 \begin{gather*}
	 	x \in L \Rightarrow \exists y: V^y(x) = 1\\
	 	x \notin L \Rightarrow \forall y: V^y(x) = 0
	 \end{gather*}
	 and $|y| = poly(|x|)$.
 \end{reminder}

\begin{definition}[Probabilistic verifier]
	Verifier has a \emph{remote access} to the proof, where proof is some string $\in \szo^{\ast}$.
	Verifier can access any position independently in constant time.
\end{definition}

\begin{definition}[PCP verifier]
	Let $L$ be a language and $q, r:\N \to \N$.
	We say that $L$  has a $r(n), q(n)$ PCP verifier if there is a \textbf{polynomial time} probabilistic algorithm $A$ which satisfies the following:
	\begin{itemize}
		\item Efficiency: on input string $x \in \szo^{\ast}$ given a random access to string $y \in \szo^{\ast}$. A uses at most $r(n)$ random bits and makes at most $q(n)$ queries to some locations in $y$.
		We denote by $A^y(x)$ a r.v. representing output of $A$ on input $x$ with access to proof $y$.
	\item Completeness: $x \in L \Rightarrow \exists y \in \szo^{\ast}: \Prob[A^y(x) = 1] = 1$
	\item Soundness: $x \notin L \Rightarrow \forall y \in \szo^{\ast}: \Prob[A^y(x) = 1] \leq \frac{1}{2}$
	\end{itemize}

	Note that the proof can be exponential, we only care about the polynomial size addressing.
\end{definition}

\begin{definition}[PCP class]
	$L \in PCP(r(n), q(n))$ if $\exists c, d > 0: L$ asks a $(c \cdot r(n), d \cdot q(n))$-PCP verifier.
\end{definition}

\begin{lemma}[PCP properties]\label{pcp_prop}
	\begin{enumerate}
		\item We may assume for a $(r, q)$-verifier that every proof $y$
			\[ |y| \leq q \cdot 2^{r(n)} \]
			as the \# of queries is bounded by $q$ and $2^{r(n)}$ possible outcomes of random access.
		\item $PCP(r(n), q(n)) \subseteq NT(2^{\bigO(r(n))} \cdot q(n))$.
			We construct an NTM which guesses the proof of size $2^{\bigO(r(n))} \cdot q(n)$.
			$M$ deterministically runs the verifies on all possible choices of the advice.
			$M$ accepts $\iff$ the verifier accepts in ALL cases.
	\end{enumerate}
\end{lemma}

\begin{theorem}[PCP]\label{pcp}
	\[ \TNP = PCP (\log n, 1) \]
\end{theorem}
\begin{proof}
	"$\subseteq$" follows from \cref{pcp_prop}.
	\[ PCP (\log n, 1) \subseteq NT(2^{\bigO(\log n)}) = \TNP \]

	"$\supseteq$" hard.
\end{proof}

\begin{example}[PCP system]
	Let GNI be pairs of non-isomorphic graphs.
	Input: $(G_n, G_i)$ graphs on $n$ vertexes.

	Claim: GNI $\in PCP (\log n, 1)$.

	The proof will be a binary string with each location being index by adjacency matrix of a graph on $n$ vertexes.
	Assume that for correct $y$:
	\[ y[H] = \threepartdef{0} {H \equiv G_0}{1} {H \equiv G_1} { \in \szo} {else} \]

	Verifies takes at random some $b \in \szo$ than randomly picks a permutation of $[n]$ which defines $H \equiv G_0$, guesses $y[H]$ and accepts $\iff y[H] = b$.

	If $x \in GNI$ then bits corresponding to $G_0$ and $G_1$ are disjoint.
	With such proof the verifier should succeed, completeness is satisfied.

	Otherwise, there are positions in $y$ which corresponds to both $G_0$ and $G_1$.
	By the definition of $y[H]$, these bits are random, so the probability of success if $\leq \frac{1}{2}$.
\end{example}

\begin{corollary}
	$\exists \rho < 1$ such that if there is an polynomial size approximation algorithm for MAX-3-SAT then $\TP = \TNP$.
\end{corollary}

\begin{definition}[qCSP]
	A $qCSP$ instance $\varphi$ is a collection $\varphi_1, \ldots, \varphi_n$ of functions (constraints)$:\szo^n \to \szo$ such that each function $\varphi_i$ depends on at most $q$ variables.

	We say that an assignment $u \in \szo^n$ satisfies constraint $\varphi_i$ if $\varphi_i(u) = 1$.
	The fraction of constraint satisfied by $u$ is $\sum \frac{\varphi_i(n)}{n}$ and let $val(\varphi_i)$ max of this value over all $u \in \szo^n$.

	Also $\varphi$ is satisfiable if $val(\varphi) = 1$.

	3-SAT is a special case of qCSP, as each clause is equivalent to a function $\varphi_i$.
\end{definition}

\begin{definition}[Gap CSP]
	$\forall q \in \N, \rho \leq 1$ a $\rho$-GAP-qCSP is a problem determining whether a given qCSP instance has $val(\varphi) = 1$ or $val(\varphi) < \rho$.

	We say that $\rho$-GAP-qCSP is $\TNP$-hard if there is a polynomial time function $f$ mapping strings to representations of qCSP satisfying
	\begin{gather*}
		x \in L \Rightarrow val(f(x)) = 1 \\
		x \notin L \Rightarrow val(f(x)) < \rho
	\end{gather*}
\end{definition}

\begin{theorem}[PCP 1]\label{pcp_1}
	There exists $\rho < 1: \forall L \in \TNP$ there is polynomial time function $f$ mapping string to representations of 3CNF:
	\begin{gather*}
		x \in L \Rightarrow val(f(x)) = 1 \\
		x \notin L \Rightarrow val(f(x)) < \rho
	\end{gather*}

	As 3SAT is $\TNP$-complete, we can always map instance of $L$ to satisfiable or unsatisfiable 3CNF.
	Theorems 2nd claim is stronger: as the standard reduction only guarantees 1 unsatisfiable clause.
	In this case, we get constant fraction of unsatisfiable clauses.
\end{theorem}
%\begin{proof}
%\end{proof}

\begin{theorem}[PCP 1 stronger]
	$\forall \varepsilon > 0, \forall \rho = \frac{7}{8} + \varepsilon: \forall L \in \TNP$ there is polynomial time function $f$ mapping string to representations of 3CNF:
	\begin{gather*}
		x \in L \Rightarrow val(f(x)) = 1 \\
		x \notin L \Rightarrow val(f(x)) < \rho
	\end{gather*}
\end{theorem}

\begin{theorem}[PCP 2]\label{pcp_2}
	$\exists q \in \N, \rho < 1: \rho$-GAP-qCSP is $\TNP$-hard.
\end{theorem}

\begin{theorem}[PCP equivalence]
	PCP \cref{pcp} is equivalent to PCP 2 \cref{pcp_2}.
\end{theorem}
\begin{proof}
	"$\Rightarrow$". Assume $ \TNP = PCP (\log n, 1)$.
	We show that $\frac{1}{2}$-GAP-qCSP is $\TNP$-hard by a reduction from 3-SAT to $\frac{1}{2}$-GAP-qCSP.
	By assumption, there is a PCP system with polynomial time verifier $V$ which makes $q$ queries (constant) and uses $c \cdot \log n$ random bits.
	Define $V_{x, r}$ function whether $V$ accepts $x$ with random coins $r$.
	Which depends on at most $q$ variables, as we make at most $q$ queries.
	Taking $\varphi$ as collection of all functions they form a $qCSP$.
	Does it satisfy GAP-qCSP?

	If we get an instance of 3-SAT which is satisfiable, then by completeness all constraints should be satisfied.

	On other hand, with a non-satisfiable instance of 3-SAT by soundness at least half of the constraints will give 0. Then $val(\varphi) \leq \frac{1}{2}$.

	Note that
	\begin{itemize}
		\item PCP verifier $V \iff CSP$ instance $\varphi$.
		\item PCP proof $y \iff$ assignment of variables in $\varphi$
		\item length of the proof $\iff$ \# of variables.
			We know: \$ of variables $\leq q \cdot 2^r = q \cdot 2^{c \log n} = q \cdot n^c$.
	\end{itemize}

	"$\Leftarrow$". Assume $ \rho$-GAP-qCSP is $\TNP$-hard for some constant $\rho, q$.
	Take $L \in \TNP$ arbitrary we want to design PCP system with parameters $(\log n, 1)$ for $L$.
	Let $f$ be a polynomial time reduction from $L \to \rho$-GAP-qCSP
	The verifier on input $x$ constructs $f(x) = \varphi$ which is a qCSP instance.
	Verifier chooses at random $i \in [n]$ and checks $\varphi_i$ by making $q$ queries.

	If $x \in L \Rightarrow \varphi$ accepts with probability 1.
	On the other hand, $x \notin L \varphi \Rightarrow$ accepts with probability $< \rho$.
	We can adjust the probability to $\frac{1}{2}$ by running the verifier polynomial times.
\end{proof}
